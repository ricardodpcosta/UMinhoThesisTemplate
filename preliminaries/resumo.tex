%=============================================================
% Ricardo Costa, December, 2021
%=============================================================
% - Title and abstract of the work in portuguese
% - At the end of the abstract, three to five keywords must be included, written in alphabetical order
% - The abstract and keywords must have a maximum length of one page
%=============================================================

\chapter*{Resumo}

A aplica\c c\~ao da modela\c c\~ao computacional em problemas de engenharia no processamento de pol\'imeros assistiu a um crescimento not\'avel nos \'ultimos anos, permitindo \`as ind\'ustrias utilizar ferramentas poderosas de conce\c c\~ao assistida por computador.
Al\'em do desenvolvimento de computadores potentes para superar as limita\c c\~oes computacionais, o desenvolvimento de m\'etodos num\'ericos proficientes e precisos tamb\'em contribuiu significativamente para a aplica\c c\~ao da modela\c c\~ao computacional em problemas de engenharia outrora intrat\'aveis.
No entanto, problemas cada vez mais complexos no processamento de pol\'imeros devido \`as geometrias comumente elaboradas, aos processos n\~ao-isot\'ermicos e fluidos polim\'ericos com comportamento reol\'ogico complexo, incrementam claramente a necessidade de maior precis\~ao num\'erica e efici\^encia computacional.

O principal objetivo do presente trabalho prende-se com a melhoria do desempenho, em termos de precis\~ao num\'erica e efici\^encia computacional, das ferramentas computacionais empregues na resolu\c c\~ao de problemas complexos na \'area do processamento de pol\'imeros.
Nesse sentido, m\'etodos num\'ericos avan\c cados s\~ao desenvolvidos no contexto do m\'etodo dos volume finitos de forma a obter uma converg\^encia do erro com o refinamento de malha maior do que as cl\'assicas primeira e segunda ordens, desta forma resultando em ganhos substanciais de precis\~ao.
Para al\'em disso, uma implementa\c c\~ao eficiente dos m\'etodos num\'ericos propostos \'e tamb\'em desenvolvida, elaborando algoritmos que reduzem o custo computacional das simula\c c\~oes e, ao mesmo tempo, tirem tamb\'em partido dos atuais processadores com capacidade de c\'alculo paralelo.

Uma an\'alise e verifica\c c\~ao aos desenvolvimentos num\'ericos e \`a implementa\c c\~ao computacional foi exaustivamente levada a cabo com casos de estudo espec\'ificos para avaliar o desempenho dos m\'etodos e algoritmos propostos.
Os resultados obtidos comprovam que os m\'etodos propostos atingem ordens de converg\^encia elevadas e \'otimas para o erro, sendo capazes de obter efetivamente, com malhas significativamente mais grosseiras, o mesmo n\'ivel de precis\~ao da solu\c c\~ao em compara\c c\~ao com os m\'etodos de primeira e segunda ordens.
Adicionalmente, ganhos substanciais em efici\^encia computacional, quer em termos de tempo de execu\c c\~ao quer dos requisitos de mem\'oria, tamb\'em s\~ao alcan\c cados, dado que os algoritmos propostos potenciam essas melhorias sem perda de precis\~ao num\'erica.

Os desenvolvimentos conseguidos representam um avan\c co importante para simula\c c\~oes mais precisas e computacionalmente mais eficientes de aplica\c c\~oes complexas no processamento de pol\'imeros.

\vspace{0.5cm}

\begin{resumopalavraschave}
Modela\c c\~ao computacional, m\'etodos num\'ericos, ordem de converg\^encia elevada, processamento de pol\'imeros, simula\c c\~oes de alta performance
\end{resumopalavraschave}

% end of file
