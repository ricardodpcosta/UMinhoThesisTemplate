%=============================================================
% Ricardo Costa, December, 2021
%=============================================================
% - Abstract of the work in a foreign language of wide usage
% - At the end of the abstract, three to five keywords must be included, written in alphabetical order
% - The abstract and keywords must have a maximum length of one page
% - Whenever the work is written in a foreign language (other than Portuguese) the same language as
% the work should be used for this abstract and keywords
%=============================================================

\chapter*{Abstract}

The application of the computational modelling in engineering problems of polymer processing has seen a remarkable growth in the past years, providing valuable computer-aided design tools to the related industries.
Besides the development of powerful hardware to overcome the computational limitations, the development of proficient and accurate numerical methods has also significantly contributed to the applicability of the computational modelling to once intractable engineering problems.
However, the ever-increasing complex polymer processing applications, comprising intricate three-dimensional geometries, non-isothermal processes, and polymeric fluids with complex rheological behaviour, clearly still raise the demand of numerical accuracy and computational efficiency.

The main objective of the present work is to improve the performance, either in terms of numerical accuracy and computational efficiency, of computational modelling tools to solve complex problems akin to polymer processing applications.
In that regard, advanced numerical methods are developed in the finite volume method context, capable of providing an error convergence under mesh refinement higher than the classical first- and second-orders, therefore resulting in substantial accuracy gains.
Moreover, the implementation efficiency of the proposed numerical methods is also addressed with algorithms that reduce the computational cost of the simulations, also taking advantage of modern multi-core processor computers.

A comprehensive analysis and verification, both of the numerical developments and the computational implementations, were exhaustively carried out with specific case studies to assess the performance of the proposed methods and algorithms.
The obtained results prove that the proposed methods achieve the optimal high-order of convergence for the error and are capable of effectively obtaining the same solution accuracy level given by lower-order ones with significantly coarser meshes.
Additionally, substantial gains in computational efficiency, both in terms of running time and memory usage, are also achieved, since the proposed algorithms further enhance these improvements without loss of numerical accuracy.

The achieved developments represent a significant advance towards more accurate and more computationally efficient simulations of complex polymer processing applications.

\vspace{0.5cm}

\begin{abstractkeywords}
Computational modelling, high-performance simulations, numerical methods, polymer processing, very high-order of convergence
\end{abstractkeywords}

% end of file
