\section{Introduction}
\label{chap2:sec:introduction}

The treatment of ever-increasing complexity problems in engineering has only been possible due to the capabilities of modern-day computational methods and high-performance computers.
In what concerns to the numerical modelling, computational efficiency is usually determined based on the computational effort necessary to obtain a certain level of solution accuracy, leading to a trade-off between the convergence order of the method and the mesh characteristic size.
In the absence of shocks or irregularities, increasing the convergence order is more efficient concerning computational resources than mesh refinement.
However, obtaining very high-order accurate approximations is still a challenging task, and many developments in that field are to be made.

The majority of the very high-order accurate methods (more than the second-order) are specifically designed for polygonal (or polyhedral) domains and, usually, numerical difficulties in obtaining the optimal convergence order arise when handling boundary conditions prescribed on curved boundaries.
For a short literature review on the topic, the reader is referred to the introduction section in R. Costa et al., 2018~\cite{chap2:2018costa1}, which is summarized in the following.
The classical approach to handle boundary conditions on curved boundaries is based on the isoparametric element method~\cite{chap2:2016lehrenfeld,chap2:2017lehrenfeld}, which requires, on one side, the introduction of curved mesh elements and, on the other side, non-linear transformations to map the local curved mesh elements onto the reference polygonal ones.
An alternative approach, dedicated to the finite volume method, was initially proposed by C. Ollivier-Gooch et al., 2002~\cite{chap2:2002ollivier}.
The technique does not require non-linear transformations, but the main shortcoming remains, in particular, the meshing algorithm to generate curved mesh elements fitting curved boundaries in addition to the high-order accurate quadrature rules for numerical integration on non-polygonal mesh elements.
As a consequence, handling arbitrary two- or three-dimensional curved elements turns out to be a cumbersome task, which results in significant computational costs~\cite{chap2:2009geuziane,chap2:2014wang,chap2:2015moxey}.

R. Costa et al., 2018~\cite{chap2:2018costa1}, introduced a new approach in the finite volume context, the reconstruction for off-site data method (shortened to ROD method), which is capable of handling boundary conditions on arbitrary smooth curved boundaries with a very high-order of convergence.
The novelty of the method is to use only polygonal mesh elements, overcoming the mismatch between the mesh boundary and the domain boundary.
The method enforces the prescribed boundary conditions using polynomial reconstructions in the vicinity of the boundary, which are computed based on the constrained least-squares method.
Moreover, the governing equations are integrated on polygonal mesh elements and, consequently, the numerical heat fluxes are determined solely on the boundaries of the polygonal cells.
Therefore, no sophisticated meshing algorithms for curved mesh elements are required, nor non-linear transformations, nor cumbersome quadrature rules for integration in the curved elements.
There are very few methods capable of handling curved domains with polygonal meshes, and most of them are limited to the first- or second-order of convergence.
Recently an extension of the immersed boundary method to the fourth-order of convergence has been proposed in the framework of the Fourier spectral method~\cite{chap2:2016stein,chap2:2017stein}, which is able of handling arbitrary smooth curved domains.

The ROD method was initially developed only for the steady-state two-dimensional convection-diffusion problem with Dirichlet boundary conditions.
In the present work, essential developments are introduced to the method, namely, the handling of Neumann and Robin boundary conditions, which represents a fundamental advance for real context applications.
Moreover, the development of a generic framework to compute polynomial reconstructions based on the least-squares method allows the handling of general constraints and improves the algorithm.

The remaining sections of the chapter are organized as follows.
Section~2 presents the model, the mesh, and the basic assumptions and notations.
Section~3 introduces the generic framework to compute polynomial reconstructions based on the least-squares method.
Section~4 is dedicated to the ROD method based on the previously introduced polynomial reconstructions and the Dirichlet, Neumann, and Robin boundary conditions on curved boundaries are addressed.
Section~5 presents the very high-order accurate finite volume scheme based on the polynomial reconstructions and the ROD method.
Section~6 provides a comprehensive numerical benchmark test suite to verify and assess the proposed method.
The chapter is completed in Section~7 with the conclusions and some perspectives for future work.

% end of file
